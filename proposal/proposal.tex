\title{CS 341 Project Proposal: \\
Predicting Ridesharing Demand}
\author{
        Ramon Iglesias \\
            \and
        Federico Rossi\\
        	\and
        Kevin Wang
}
\date{\today}

\documentclass[12pt]{article}

\begin{document}
\maketitle

\section{Problem Description}

Personal mobility in the form of privately owned automobiles contributes to increasing levels of traffic congestion, pollution, and under-utilization of vehicles (on average 5\% in the US \cite{DN:15}) -- clearly unsustainable trends for the future. The pressing need to reverse these trends has spurred the creation of cost competitive, on-demand personal mobility solutions such as car-sharing (e.g. Car2Go, ZipCar) and ride-sharing (e.g. Uber, Lyft).
However, without proper fleet management, car-sharing and ride-sharing systems lead to vehicle imbalances: vehicles aggregate in some areas while becoming depleted in others, due to the asymmetry between trip origins and destinations \cite{RZ-MP:15_MODa}. This issue has been addressed in a variety of ways in the literature. The works in \cite{MN-SZ-SB-MJR:15}, \cite{BB-KGZ-NG:15}, and \cite{FA-DDP-AR:14} investigate using paid drivers to move vehicles between car-sharing stations where cars are parked, while \cite{SB-RJ-CR:15} studies the merits of dynamic pricing for incentivizing drivers to move to underserved areas. Moreover, \cite{RZ-MP:15_MODa,RI-FR-RZ-MP:16,RZ-FR-MP:16,RZ-FR-MP:16a} tackle the problem under the assumption of \emph{autonomous} vehicles, and cast the problem from the perspective of optimal control.

Nonetheless, these approaches are reactive, that is, they react to current state conditions rather predicted future conditions, e.g. future customer demand. We believe that robust and accurate customer demand prediction models would enable a new family of car- and ride-sharing rebalancing problems. Thus, we propose the development of large-scale predictive models that provide mobility demand prediction at different time horizons.

\section{Data}


\section{Methodology}\label{methods}


\section{Evaluation}\label{evaluation}


\section{Plan}\label{plan}


\section{About Us}\label{about}

\bibliographystyle{abbrv}
\bibliography{main}

\end{document}
This is never printed